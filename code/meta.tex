% Options for packages loaded elsewhere
\PassOptionsToPackage{unicode}{hyperref}
\PassOptionsToPackage{hyphens}{url}
%
\documentclass[
]{article}
\usepackage{amsmath,amssymb}
\usepackage{iftex}
\ifPDFTeX
  \usepackage[T1]{fontenc}
  \usepackage[utf8]{inputenc}
  \usepackage{textcomp} % provide euro and other symbols
\else % if luatex or xetex
  \usepackage{unicode-math} % this also loads fontspec
  \defaultfontfeatures{Scale=MatchLowercase}
  \defaultfontfeatures[\rmfamily]{Ligatures=TeX,Scale=1}
\fi
\usepackage{lmodern}
\ifPDFTeX\else
  % xetex/luatex font selection
\fi
% Use upquote if available, for straight quotes in verbatim environments
\IfFileExists{upquote.sty}{\usepackage{upquote}}{}
\IfFileExists{microtype.sty}{% use microtype if available
  \usepackage[]{microtype}
  \UseMicrotypeSet[protrusion]{basicmath} % disable protrusion for tt fonts
}{}
\makeatletter
\@ifundefined{KOMAClassName}{% if non-KOMA class
  \IfFileExists{parskip.sty}{%
    \usepackage{parskip}
  }{% else
    \setlength{\parindent}{0pt}
    \setlength{\parskip}{6pt plus 2pt minus 1pt}}
}{% if KOMA class
  \KOMAoptions{parskip=half}}
\makeatother
\usepackage{xcolor}
\usepackage[margin=1in]{geometry}
\usepackage{color}
\usepackage{fancyvrb}
\newcommand{\VerbBar}{|}
\newcommand{\VERB}{\Verb[commandchars=\\\{\}]}
\DefineVerbatimEnvironment{Highlighting}{Verbatim}{commandchars=\\\{\}}
% Add ',fontsize=\small' for more characters per line
\usepackage{framed}
\definecolor{shadecolor}{RGB}{248,248,248}
\newenvironment{Shaded}{\begin{snugshade}}{\end{snugshade}}
\newcommand{\AlertTok}[1]{\textcolor[rgb]{0.94,0.16,0.16}{#1}}
\newcommand{\AnnotationTok}[1]{\textcolor[rgb]{0.56,0.35,0.01}{\textbf{\textit{#1}}}}
\newcommand{\AttributeTok}[1]{\textcolor[rgb]{0.13,0.29,0.53}{#1}}
\newcommand{\BaseNTok}[1]{\textcolor[rgb]{0.00,0.00,0.81}{#1}}
\newcommand{\BuiltInTok}[1]{#1}
\newcommand{\CharTok}[1]{\textcolor[rgb]{0.31,0.60,0.02}{#1}}
\newcommand{\CommentTok}[1]{\textcolor[rgb]{0.56,0.35,0.01}{\textit{#1}}}
\newcommand{\CommentVarTok}[1]{\textcolor[rgb]{0.56,0.35,0.01}{\textbf{\textit{#1}}}}
\newcommand{\ConstantTok}[1]{\textcolor[rgb]{0.56,0.35,0.01}{#1}}
\newcommand{\ControlFlowTok}[1]{\textcolor[rgb]{0.13,0.29,0.53}{\textbf{#1}}}
\newcommand{\DataTypeTok}[1]{\textcolor[rgb]{0.13,0.29,0.53}{#1}}
\newcommand{\DecValTok}[1]{\textcolor[rgb]{0.00,0.00,0.81}{#1}}
\newcommand{\DocumentationTok}[1]{\textcolor[rgb]{0.56,0.35,0.01}{\textbf{\textit{#1}}}}
\newcommand{\ErrorTok}[1]{\textcolor[rgb]{0.64,0.00,0.00}{\textbf{#1}}}
\newcommand{\ExtensionTok}[1]{#1}
\newcommand{\FloatTok}[1]{\textcolor[rgb]{0.00,0.00,0.81}{#1}}
\newcommand{\FunctionTok}[1]{\textcolor[rgb]{0.13,0.29,0.53}{\textbf{#1}}}
\newcommand{\ImportTok}[1]{#1}
\newcommand{\InformationTok}[1]{\textcolor[rgb]{0.56,0.35,0.01}{\textbf{\textit{#1}}}}
\newcommand{\KeywordTok}[1]{\textcolor[rgb]{0.13,0.29,0.53}{\textbf{#1}}}
\newcommand{\NormalTok}[1]{#1}
\newcommand{\OperatorTok}[1]{\textcolor[rgb]{0.81,0.36,0.00}{\textbf{#1}}}
\newcommand{\OtherTok}[1]{\textcolor[rgb]{0.56,0.35,0.01}{#1}}
\newcommand{\PreprocessorTok}[1]{\textcolor[rgb]{0.56,0.35,0.01}{\textit{#1}}}
\newcommand{\RegionMarkerTok}[1]{#1}
\newcommand{\SpecialCharTok}[1]{\textcolor[rgb]{0.81,0.36,0.00}{\textbf{#1}}}
\newcommand{\SpecialStringTok}[1]{\textcolor[rgb]{0.31,0.60,0.02}{#1}}
\newcommand{\StringTok}[1]{\textcolor[rgb]{0.31,0.60,0.02}{#1}}
\newcommand{\VariableTok}[1]{\textcolor[rgb]{0.00,0.00,0.00}{#1}}
\newcommand{\VerbatimStringTok}[1]{\textcolor[rgb]{0.31,0.60,0.02}{#1}}
\newcommand{\WarningTok}[1]{\textcolor[rgb]{0.56,0.35,0.01}{\textbf{\textit{#1}}}}
\usepackage{graphicx}
\makeatletter
\def\maxwidth{\ifdim\Gin@nat@width>\linewidth\linewidth\else\Gin@nat@width\fi}
\def\maxheight{\ifdim\Gin@nat@height>\textheight\textheight\else\Gin@nat@height\fi}
\makeatother
% Scale images if necessary, so that they will not overflow the page
% margins by default, and it is still possible to overwrite the defaults
% using explicit options in \includegraphics[width, height, ...]{}
\setkeys{Gin}{width=\maxwidth,height=\maxheight,keepaspectratio}
% Set default figure placement to htbp
\makeatletter
\def\fps@figure{htbp}
\makeatother
\setlength{\emergencystretch}{3em} % prevent overfull lines
\providecommand{\tightlist}{%
  \setlength{\itemsep}{0pt}\setlength{\parskip}{0pt}}
\setcounter{secnumdepth}{-\maxdimen} % remove section numbering
\ifLuaTeX
  \usepackage{selnolig}  % disable illegal ligatures
\fi
\usepackage{bookmark}
\IfFileExists{xurl.sty}{\usepackage{xurl}}{} % add URL line breaks if available
\urlstyle{same}
\hypersetup{
  pdftitle={meta},
  hidelinks,
  pdfcreator={LaTeX via pandoc}}

\title{meta}
\author{}
\date{\vspace{-2.5em}2024-05-30}

\begin{document}
\maketitle

\subsection{R Markdown}\label{r-markdown}

\subsection{blabla}\label{blabla}

Partimos de dos datasets, donde tenemos el total de minutos, partidos,
puntos, rebotes, asistencias, tapones y robos efectuados por los
jugadores de la NBA y las jugadores de la WNBA, que son la liga
masuclina y femenina de baloncesto de Estados Unidos. Estos datos
corresponden a la temporada 2016-17.

Nuestro objetivo es unificar estas fuentes en un unico dataset,
limpiarlo, normalizarlo si es necesario, y establecer visualizaciones
que nos permitan obtener información sobre la brecha salarial existente
entre ambas ligas.

\begin{Shaded}
\begin{Highlighting}[]
\CommentTok{\# read data}
\NormalTok{films }\OtherTok{\textless{}{-}} \FunctionTok{read.csv}\NormalTok{(}\StringTok{"../data/films.csv"}\NormalTok{ ,}\AttributeTok{sep=}\StringTok{";"}\NormalTok{)}

\FunctionTok{head}\NormalTok{(films)}
\end{Highlighting}
\end{Shaded}

\begin{verbatim}
##                      title year    rating                         studio
## 1           Dekalog (1988) 1996     TV-MA Facets Multimedia Distribution
## 2            The Godfather 1972         R             Paramount Pictures
## 3              Tokyo Story 1972 Not Rated               New Yorker Films
## 4             Citizen Kane 1941  Approved             Paramount Pictures
## 5                  Boyhood 2014         R                      IFC Films
## 6 The Leopard (re-release) 2004  Approved          Twentieth Century Fox
##   duration must_see metascore user_score number_critics number_users
## 1 9 h 32 m    False       100        7.5             13          100
## 2 2 h 55 m     True       100        9.3             16            3
## 3 2 h 16 m     True       100        8.2             19          132
## 4 1 h 59 m     True       100        8.3             19          875
## 5 2 h 45 m     True       100        7.5             50            2
## 6  3 h 6 m    False       100        7.9             12           77
##            genre             director
## 1          Drama Krzysztof Kieslowski
## 2   Crime, Drama Francis Ford Coppola
## 3          Drama         Yasujirô Ozu
## 4 Drama, Mystery         Orson Welles
## 5          Drama    Richard Linklater
## 6 Drama, History     Luchino Visconti
##                                                                                                                             writer
## 1                                                                                       Krzysztof Kieslowski, Krzysztof Piesiewicz
## 2                                                                                                 Mario Puzo, Francis Ford Coppola
## 3                                                                                                          Kôgo Noda, Yasujirô Ozu
## 4                                                   Herman J. Mankiewicz, John Houseman, Roger Q. Denny, Mollie Kent, Orson Welles
## 5                                                                                                                Richard Linklater
## 6 Giuseppe Tomasi di Lampedusa, Suso Cecchi D'Amico, Pasquale Festa Campanile, Enrico Medioli, Massimo Franciosa, Luchino Visconti
##                                                                                                                                                                                                                                                                                                                                                                                                                                                                                                                                                                                                                                                                                                                                                                                                                                                                                                                                                                                                                           summary
## 1 This masterwork by Krzysztof Kieślowski is one of the twentieth century’s greatest achievements in visual storytelling. Originally made for Polish television, Dekalog focuses on the residents of a housing complex in late-Communist Poland, whose lives become subtly intertwined as they face emotional dilemmas that are at once deeply personal and universally human. Its ten hour-long films, drawing from the Ten Commandments for thematic inspiration and an overarching structure, grapple deftly with complex moral and existential questions concerning life, death, love, hate, truth, and the passage of time. Shot by nine different cinematographers, with stirring music by Zbigniew Preisner and compelling performances from established and unknown actors alike, Dekalog arrestingly explores the unknowable forces that shape our lives. Also available are the longer theatrical versions of the series’ fifth and sixth films: A Short Film About Killing and A Short Film About Love. [Janus Films].
## 2                                                                                                                                                                                                                                                                                                                                                                                          Francis Ford Coppola's epic features Marlon Brando in his Oscar-winning role as the patriarch of the Corleone family. Director Coppola paints a chilling portrait of the Sicilian clan's rise and near fall from power in America, masterfully balancing the story between the Corleone's family life and the ugly crime business in which they are engaged. Based on Mario Puzo's best-selling novel and featuring career-making performances by Al Pacino, James Caan and Robert Duvall, this searing and brilliant film garnered ten Academy Award nominations, and won three including Best Picture of 1972. [Paramount Pictures].
## 3                                                                                                                                                                                                                                    Yasujiro Ozu’s Tokyo Story follows an aging couple, Tomi and Sukichi, on their journey from their rural village to visit their two married children in bustling, postwar Tokyo. Their reception is disappointing: too busy to entertain them, their children send them off to a health spa. After Tomi falls ill she and Sukichi return home, while the children, grief-stricken, hasten to be with her. From a simple tale unfolds one of the greatest of all Japanese films. Starring Ozu regulars Chishu Ryu and Setsuko Hara, the film reprises one of the director’s favorite themes—that of generational conflict—in a way that is quintessentially Japanese and yet so universal in its appeal that it continues to resonate as one of cinema’s greatest masterpieces. [Janus Films].
## 4                                                                                                                                                                                                                                                                                                                                                                                                                                                                                                                                                                                                                                                                                                                                                                                                                                                                                                             Following the death of a publishing tycoon, news reporters scramble to discover the meaning of his final utterance.
## 5                                                                                                                                                                                                                                                                                                                                                                                                                                                                                  Filmed over 12 years with the same cast, Richard Linklater's Boyhood is a groundbreaking story of growing up as seen through the eyes of a child named Mason (Ellar Coltrane), who literally grows up on screen before our eyes. Starring Ethan Hawke and Patricia Arquette as Mason's parents and newcomer Lorelei Linklater as his sister Samantha, Boyhood charts the rocky terrain of childhood like no other film has before and is both a nostalgic time capsule of the recent past and an ode to growing up and parenting. [IFC Films].
## 6                                                                                                                                                                                                                                                                                                                                                                                                                                                                                                                                                                                                                                                                                                                                                                              Set in Sicily in 1860, Luchino Visconti's spectacular 1963 adaptation of Giuseppe di Lampedusa's international bestseller is one of the cinema's greatest evocations of the past, achingly depicting the passing of an ancient order. (Film Forum)
\end{verbatim}

\begin{Shaded}
\begin{Highlighting}[]
\FunctionTok{summary}\NormalTok{(films)}
\end{Highlighting}
\end{Shaded}

\begin{verbatim}
##     title                year         rating             studio         
##  Length:2596        Min.   :1916   Length:2596        Length:2596       
##  Class :character   1st Qu.:1996   Class :character   Class :character  
##  Mode  :character   Median :2010   Mode  :character   Mode  :character  
##                     Mean   :2003                                        
##                     3rd Qu.:2018                                        
##                     Max.   :2024                                        
##                     NA's   :24                                          
##    duration           must_see           metascore       user_score       
##  Length:2596        Length:2596        Min.   : 77.00   Length:2596       
##  Class :character   Class :character   1st Qu.: 79.00   Class :character  
##  Mode  :character   Mode  :character   Median : 82.00   Mode  :character  
##                                        Mean   : 83.44                     
##                                        3rd Qu.: 86.00                     
##                                        Max.   :100.00                     
##                                                                           
##  number_critics   number_users      genre             director        
##  Min.   : 7.00   Min.   :  1.0   Length:2596        Length:2596       
##  1st Qu.:13.00   1st Qu.: 11.0   Class :character   Class :character  
##  Median :19.00   Median : 31.0   Mode  :character   Mode  :character  
##  Mean   :22.45   Mean   :105.9                                        
##  3rd Qu.:30.00   3rd Qu.:111.0                                        
##  Max.   :69.00   Max.   :997.0                                        
##                  NA's   :488                                          
##     writer            summary         
##  Length:2596        Length:2596       
##  Class :character   Class :character  
##  Mode  :character   Mode  :character  
##                                       
##                                       
##                                       
## 
\end{verbatim}

El dataset y el código del mismo esta localizable en la siguiente
dirección:

\url{https://github.com/abarreraquintanilla/metacritic_cleaning.git}

\begin{center}\rule{0.5\linewidth}{0.5pt}\end{center}

\section{Limpieza de datos}\label{limpieza-de-datos}

\begin{center}\rule{0.5\linewidth}{0.5pt}\end{center}

El fichero de datos contiene 2596 registros y 14 variables.

`r toString(n.var)

\begin{Shaded}
\begin{Highlighting}[]
\NormalTok{films}\OtherTok{\textless{}{-}}\NormalTok{films[,}\SpecialCharTok{{-}}\DecValTok{14}\NormalTok{]}
\FunctionTok{summary}\NormalTok{(films)}
\end{Highlighting}
\end{Shaded}

\begin{verbatim}
##     title                year         rating             studio         
##  Length:2596        Min.   :1916   Length:2596        Length:2596       
##  Class :character   1st Qu.:1996   Class :character   Class :character  
##  Mode  :character   Median :2010   Mode  :character   Mode  :character  
##                     Mean   :2003                                        
##                     3rd Qu.:2018                                        
##                     Max.   :2024                                        
##                     NA's   :24                                          
##    duration           must_see           metascore       user_score       
##  Length:2596        Length:2596        Min.   : 77.00   Length:2596       
##  Class :character   Class :character   1st Qu.: 79.00   Class :character  
##  Mode  :character   Mode  :character   Median : 82.00   Mode  :character  
##                                        Mean   : 83.44                     
##                                        3rd Qu.: 86.00                     
##                                        Max.   :100.00                     
##                                                                           
##  number_critics   number_users      genre             director        
##  Min.   : 7.00   Min.   :  1.0   Length:2596        Length:2596       
##  1st Qu.:13.00   1st Qu.: 11.0   Class :character   Class :character  
##  Median :19.00   Median : 31.0   Mode  :character   Mode  :character  
##  Mean   :22.45   Mean   :105.9                                        
##  3rd Qu.:30.00   3rd Qu.:111.0                                        
##  Max.   :69.00   Max.   :997.0                                        
##                  NA's   :488                                          
##     writer         
##  Length:2596       
##  Class :character  
##  Mode  :character  
##                    
##                    
##                    
## 
\end{verbatim}

\end{document}
